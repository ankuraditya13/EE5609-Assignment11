\documentclass[journal,12pt,twocolumn]{IEEEtran}

\usepackage{setspace}
\usepackage{gensymb}

\singlespacing


\usepackage[cmex10]{amsmath}

\usepackage{amsthm}

\usepackage{mathrsfs}
\usepackage{txfonts}
\usepackage{stfloats}
\usepackage{bm}
\usepackage{cite}
\usepackage{cases}
\usepackage{subfig}

\usepackage{longtable}
\usepackage{multirow}

\usepackage{enumitem}
\usepackage{mathtools}
\usepackage{steinmetz}
\usepackage{tikz}
\usepackage{circuitikz}
\usepackage{verbatim}
\usepackage{tfrupee}
\usepackage[breaklinks=true]{hyperref}
\usepackage{graphicx}
\usepackage{tkz-euclide}

\usetikzlibrary{calc,math}
\usepackage{listings}
    \usepackage{color}                                            %%
    \usepackage{array}                                            %%
    \usepackage{longtable}                                        %%
    \usepackage{calc}                                             %%
    \usepackage{multirow}                                         %%
    \usepackage{hhline}                                           %%
    \usepackage{ifthen}                                           %%
    \usepackage{lscape}     
\usepackage{multicol}
\usepackage{chngcntr}

\DeclareMathOperator*{\Res}{Res}

\renewcommand\thesection{\arabic{section}}
\renewcommand\thesubsection{\thesection.\arabic{subsection}}
\renewcommand\thesubsubsection{\thesubsection.\arabic{subsubsection}}

\renewcommand\thesectiondis{\arabic{section}}
\renewcommand\thesubsectiondis{\thesectiondis.\arabic{subsection}}
\renewcommand\thesubsubsectiondis{\thesubsectiondis.\arabic{subsubsection}}


\hyphenation{op-tical net-works semi-conduc-tor}
\def\inputGnumericTable{}                                 %%

\lstset{
%language=C,
frame=single, 
breaklines=true,
columns=fullflexible
}
\begin{document}


\newtheorem{theorem}{Theorem}[section]
\newtheorem{problem}{Problem}
\newtheorem{proposition}{Proposition}[section]
\newtheorem{lemma}{Lemma}[section]
\newtheorem{corollary}[theorem]{Corollary}
\newtheorem{example}{Example}[section]
\newtheorem{definition}[problem]{Definition}

\newcommand{\BEQA}{\begin{eqnarray}}
\newcommand{\EEQA}{\end{eqnarray}}
\newcommand{\define}{\stackrel{\triangle}{=}}
\bibliographystyle{IEEEtran}
\providecommand{\mbf}{\mathbf}
\providecommand{\pr}[1]{\ensuremath{\Pr\left(#1\right)}}
\providecommand{\qfunc}[1]{\ensuremath{Q\left(#1\right)}}
\providecommand{\sbrak}[1]{\ensuremath{{}\left[#1\right]}}
\providecommand{\lsbrak}[1]{\ensuremath{{}\left[#1\right.}}
\providecommand{\rsbrak}[1]{\ensuremath{{}\left.#1\right]}}
\providecommand{\brak}[1]{\ensuremath{\left(#1\right)}}
\providecommand{\lbrak}[1]{\ensuremath{\left(#1\right.}}
\providecommand{\rbrak}[1]{\ensuremath{\left.#1\right)}}
\providecommand{\cbrak}[1]{\ensuremath{\left\{#1\right\}}}
\providecommand{\lcbrak}[1]{\ensuremath{\left\{#1\right.}}
\providecommand{\rcbrak}[1]{\ensuremath{\left.#1\right\}}}
\theoremstyle{remark}
\newtheorem{rem}{Remark}
\newcommand{\sgn}{\mathop{\mathrm{sgn}}}
\providecommand{\abs}[1]{\left\vert#1\right\vert}
\providecommand{\res}[1]{\Res\displaylimits_{#1}} 
\providecommand{\norm}[1]{\left\lVert#1\right\rVert}
%\providecommand{\norm}[1]{\lVert#1\rVert}
\providecommand{\mtx}[1]{\mathbf{#1}}
\providecommand{\mean}[1]{E\left[ #1 \right]}
\providecommand{\fourier}{\overset{\mathcal{F}}{ \rightleftharpoons}}
%\providecommand{\hilbert}{\overset{\mathcal{H}}{ \rightleftharpoons}}
\providecommand{\system}{\overset{\mathcal{H}}{ \longleftrightarrow}}
	%\newcommand{\solution}[2]{\textbf{Solution:}{#1}}
\newcommand{\solution}{\noindent \textbf{Solution: }}
\newcommand{\cosec}{\,\text{cosec}\,}
\providecommand{\dec}[2]{\ensuremath{\overset{#1}{\underset{#2}{\gtrless}}}}
\newcommand{\myvec}[1]{\ensuremath{\begin{pmatrix}#1\end{pmatrix}}}
\newcommand{\mydet}[1]{\ensuremath{\begin{vmatrix}#1\end{vmatrix}}}
\numberwithin{equation}{subsection}
\makeatletter
\@addtoreset{figure}{problem}
\makeatother
\let\StandardTheFigure\thefigure
\let\vec\mathbf
\renewcommand{\thefigure}{\theproblem}
\def\putbox#1#2#3{\makebox[0in][l]{\makebox[#1][l]{}\raisebox{\baselineskip}[0in][0in]{\raisebox{#2}[0in][0in]{#3}}}}
     \def\rightbox#1{\makebox[0in][r]{#1}}
     \def\centbox#1{\makebox[0in]{#1}}
     \def\topbox#1{\raisebox{-\baselineskip}[0in][0in]{#1}}
     \def\midbox#1{\raisebox{-0.5\baselineskip}[0in][0in]{#1}}
\vspace{3cm}
\title{Assignment-11}
\author{Ankur Aditya - EE20RESCH11010}
\maketitle
\newpage
\bigskip
\renewcommand{\thefigure}{\theenumi}
\renewcommand{\thetable}{\theenumi}

\begin{abstract}
This document contains the the solution of problem related to subspaces.(Hoffman Page-40, Question-5)  
\end{abstract}
Download latex-file codes from 
\begin{lstlisting}
https://github.com/ankuraditya13/EE5609-Assignment11
\end{lstlisting}

\section{Problem}
Let $\vec{F}$ be a field and let n be a positive integer (n$\geq$2). Let $\vec{V}$ be the vector space of all n$\times$n matrices over $\vec{F}$. Which of the following set of matrices $\vec{A}$ in $\vec{V}$ are subspaces of   $\vec{V}$?
\begin{enumerate}
\item all invertible $\vec{A}$;
\item all non-invertible $\vec{A}$;
\item all $\vec{A}$ such that $\vec{AB}=\vec{BA}$, where $\vec{B}$ is some fixed matrix in $\vec{V}$;
\item all $\vec{A}$ such that $\vec{A}^2 = \vec{A}$.
\end{enumerate}
\section{Solution 1}
Let the matrices $\vec{A}$ and $\vec{B}$ $\in$ $\vec{V}$, be set of invertible matrix. For them to be a subspace they need to be closed under addition.
Let,
\begin{align}
\vec{A} = \vec{I}\\
\vec{B} = -\vec{I}
\end{align}  
It could be easily proven that both matrices $\vec{A}$ and $\vec{B}$ are invertible as their determinant $\neq$ 0. Now,
\begin{align}
\vec{A}+\vec{B} = \vec{0}. 
\end{align} 
Now the zero matrix $\vec{0}$ is non-invertible as,
\begin{align}
rank(\vec{0}_{nxn}) = rank\brak{\myvec{0&0&\cdots&0\\0&0&\cdots&0\\\vdots&\vdots&\ddots&\vdots\\0&0&\cdots&0}_{nxn}}= 0
\end{align} 
\textbf{$\therefore$ the set of invertible matrices are not closed under addition. Hence not a subspace of $\vec{V}$.}
\section{Solution 2}
Let the matrices $\vec{A_1}$, $\vec{A_2}$.....$\vec{A_n}$ $\in \vec{V}$, be set of non-invertible matrix. For them to be a subspace they need to be closed under addition. Let,
\begin{align}
\vec{A_1} = \myvec{1&0&\cdots&0\\0&0&\cdots&0\\\vdots&\vdots&\vdots&\vdots\\0&0&\cdots&0}_{nxn}\\
\vec{A_2} = \myvec{0&0&\cdots&0\\0&1&\cdots&0\\0&0&\cdots&0\\\vdots&\vdots&\vdots&\vdots\\0&0&\cdots&0}_{nxn}\\
\vec{A_n}=\myvec{0&0&\cdots&0\\0&0&\cdots&0\\\vdots&\vdots&\vdots&\vdots\\0&0&\cdots&1}_{nxn}\\
\end{align}  
It could be easily proven that matrices $\vec{A_1}$, $\vec{A_2}$.....$\vec{A_n}$ are non-invertible as their determinant = 0. Now,
\begin{align}
\vec{A_1}+\vec{A_2}+\vec{A_3}+\cdots\vec{A_n} = \vec{I}_{nxn}
\end{align}
Now the identity matrix $\vec{I}$ is invertible as,
\begin{align}
rank(\vec{I}_{nxn}) = rank\brak{\myvec{1&0&\cdots&0\\0&1&\cdots&0\\\vdots&\vdots&\ddots&\vdots\\0&0&\cdots&1}_{nxn}}= n
\end{align} 
or it is a full rank matrix as there are n pivots. \textbf{$\therefore$ the set of non-invertible matrices are not closed under addition. Hence not a subspace of $\vec{V}$.}
\section{Solution 3}
\textbf{Theorem 1:}. A non-empty subset W of V is a subspace of V if and only if for each pair of vectors $\alpha$, $\beta$ in W and each scalar c $\in$ F, the vector $c\alpha+\beta \in$ W. \\
Let the matrices $\vec{A_1}$ and $\vec{A_2}$ satisfy,
\begin{align}
\vec{A_1B}=\vec{BA_1}\label{s1}\\
\vec{A_2B}=\vec{BA_2}\label{s2}
\end{align}
Let, c$\in \vec{F}$ be any constant.   
\begin{align}
\therefore \brak{c\vec{A_1}+\vec{A_2}}\vec{B} = c\vec{A_1B} +\vec{A_2B}\label{s3}
\end{align}
Substituting from equations \eqref{s1} and \eqref{s2} to \eqref{s3},
\begin{align}
\implies \brak{c\vec{A_1}+\vec{A_2}}\vec{B} = c\vec{BA_1} +\vec{BA_2}\\
\implies \vec{B}c\vec{A_1} +\vec{BA_2}\\
\implies \vec{B}\brak{c\vec{A_1}+\vec{A_2}}
\end{align}
\textbf{Thus, $\brak{c\vec{A_1}+\vec{A_2}}$ satisfy the criteria and from Theorem-1 it can be seen that the set is a subspace.} 
\section{Solution 4}
Let $\vec{A}$ and $\vec{B} \in \vec{V}$ be set of matrices such that,
\begin{align}
\vec{A^2}=\vec{A}\\
\vec{B^2}=\vec{B}
\end{align}
Now for them to be closed under addition,
\begin{align}
\brak{\vec{A+B}}^2 = \vec{A+B}
\end{align} 
Which is not always same. Example let,
\begin{align}
\vec{A} = \myvec{1&1\\0&0}\\
\vec{B} = \myvec{0&0\\0&1}
\end{align}
Clearly,
\begin{align}
\vec{A}^2 = \myvec{1&1\\0&0}\myvec{1&1\\0&0} = \myvec{1&1\\0&0} = \vec{A}\\
\vec{B}^2 =\myvec{0&0\\0&1}\myvec{0&0\\0&1} = \myvec{0&0\\0&1} = \vec{B}
\end{align}
Now, 
\begin{align}
\vec{A+B} = \myvec{1&1\\0&1}\label{e1}\\
\implies \brak{\vec{A+B}}^2 = \myvec{1&1\\0&1}\myvec{1&1\\0&1} = \myvec{1&2\\0&1}\label{e2}
\end{align}
Hence, clearly from equations \eqref{e1} and \eqref{e2},
\begin{align}
\brak{\vec{A+B}}^2 \neq \vec{A+B}
\end{align}
\textbf{$\therefore$ the set of all $\vec{A}$ such that $\vec{A}^2=\vec{A}$ is not closed under addition. Hence not a subspace in $\vec{V}$.}
\end{document}